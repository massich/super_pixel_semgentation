%  article.tex (Version 3.3, released 19 January 2008)
%  Article to demonstrate format for SPIE Proceedings
%  Special instructions are included in this file after the
%  symbol %>>>>
%  Numerous commands are commented out, but included to show how
%  to effect various options, e.g., to print page numbers, etc.
%  This LaTeX source file is composed for LaTeX2e.

%  The following commands have been added in the SPIE class
%  file (spie.cls) and will not be understood in other classes:
%  \supit{}, \authorinfo{}, \skiplinehalf, \keywords{}
%  The bibliography style file is called spiebib.bst,
%  which replaces the standard style unstr.bst.

\documentclass[letter]{spie}  %>>> use for US letter paper
%%\documentclass[a4paper]{spie}  %>>> use this instead for A4 paper
%%\documentclass[nocompress]{spie}  %>>> to avoid compression of citations
%% \addtolength{\voffset}{9mm}   %>>> moves text field down
%% \renewcommand{\baselinestretch}{1.65}   %>>> 1.65 for double spacing, 1.25 for 1.5 spacing

%% Latex documents that need direct input

%  The following command loads a graphics package to include images 
%  in the document. It may be necessary to specify a DVI driver option,
%  e.g., [dvips], but that may be inappropriate for some LaTeX 
%  installations. 
\usepackage[]{graphicx}

% Clever cross referencing. Using cleverref, instead of writting 
% figure~\ref{...} or equation~\ref{...}, only \cref{...} is required.
% The package interprates the references and introduces the figure, fig.,
% equation, eq., etc keywords. \Cref forces first letter capital. 
\usepackage{cleveref}

% In order to include files without having a clear page using \include*, 
% the newclude package is required
\usepackage{newclude}

% Required for acronyms
\usepackage{acro}

% Managing TODOES and unfinished figures
\usepackage{todonotes}

% Mathematics extra symols and commands
\usepackage{amssymb, amsmath}
        % contains the latex packages

%\title{Breast lesion Segmentation in \acl*{us} images based on super-pixels and high-level descriptors} 
\title{\acl{bus} image segmentation: an optimization approach based on super-pixels and high-level descriptors}

%>>>> The author is responsible for formatting the 
%  author list and their institutions.  Use  \skiplinehalf 
%  to separate author list from addresses and between each address.
%  The correspondence between each author and his/her address
%  can be indicated with a superscript in italics, 
%  which is easily obtained with \supit{}.

\author{Joan~Massich\supit{a} and
        Guillaume~Lema\^{i}tre\supit{a,b} and
        Joan~Mart\'{i}\supit{b} and
        Fabrice~M\'{eriaudeau}\supit{a}
  \skiplinehalf
  \supit{a}{\scriptsize LE2I-UMR CNRS 6306, Universit\'{e} de Bourgogne, 12 rue de la Fonderie, 71200 Le Creusot, France;} \\
  \supit{b}{\scriptsize ViCOROB, Universitat de Girona, Campus Montilivi, Edifici P4, 17071 Girona, Spain}
}

%>>>> Further information about the authors, other than their 
%  institution and addresses, should be included as a footnote, 
%  which is facilitated by the \authorinfo{} command.

\authorinfo{This work was partially supported by the Regional Council of Burgundy.\\Further author information, send correspondence to joan.massich@u-bourgogne.fr}
%%>>>> when using amstex, you need to use @@ instead of @
             % contains the Title and Autor info
%%%%%%%%%%%%%%%%%%%%%%%%%%%%%%%%%%%%%%%%%%%%%%%%%%%%%%%%%%%%% 
%>>>> uncomment following for page numbers
% \pagestyle{plain}    
%>>>> uncomment following to start page numbering at 301 
%\setcounter{page}{301} 
      % contains package and variables init.
%% Acronym definition example using glossaries package
%% \usepackage{acro} is required
%% 
%% For a powerful usage of the acro package look at http://tex.stackexchange.com/questions/135975/how-to-define-an-acronym-by-using-other-acronym-and-print-the-abbreviations-toge

\DeclareAcronym{us}{
  short = US,
  long  = Ultra-Sound
}

\DeclareAcronym{cad}{
  short = CAD,
  long  = Computer Aided Diagnosis
}

\DeclareAcronym{dm}{
  short = DM,
  long  = Digital Mammography
}

\DeclareAcronym{gt}{
  short = GT,
  long  = Ground Truth
}

\DeclareAcronym{bus}{
%  short = B\acs*{us},
%  long  = Breast \acifused{us}{\acs*{us}}{\acl*{us}}
short = BUS,
long= Breast Ultra-Sound
}

\DeclareAcronym{ml}{
  short = ML,
  long  = Machine Learning
}

\DeclareAcronym{svm}{
  short = SVM,
  long  = Support Vector Machine
}

\DeclareAcronym{acm}{
  short = ACM,
  long  = Active Contour Model
}

\DeclareAcronym{crf}{
  short = CRFs,
  long  = Conditional Random Fields
}

\DeclareAcronym{mrf}{
  short = MRFs,
  long  = Markov Random Fields
}

\DeclareAcronym{cv}{
  short = CV,
  long  = Computer Vision
}
\DeclareAcronym{icm}{
  short = ICM,
  long  = Iterated Conditional Modes
}
\DeclareAcronym{sa}{
  short = SA,
  long  = Simulate Anealing
}
\DeclareAcronym{gc}{
  short = GC,
  long  = Graph-Cuts
}

\DeclareAcronym{aov}{
  short = AOV,
  long  = Area Overlap
}

\DeclareAcronym{birads}{
  short = BI-RADS,
  long  = Breast Imaging-Reporting and Data System
}

\DeclareAcronym{mad}{
  short = MAD,
  long  = Median Absolute Deviation
}

\DeclareAcronym{qc}{
  short = QC,
  long  = Quadratic-Chi
}

\DeclareAcronym{sift}{
  short = SIFT,
  long  = Self-Invariant Feature Transform
}

\DeclareAcronym{bof}{
  short = BoF,
  long  = Back-of-Features
}

\DeclareAcronym{acr}{
  short = ACR,
  long  = American College of Radiology
}

\DeclareAcronym{fa}{
  short = FA,
  long  = Fibro-Adenoma
}

\DeclareAcronym{dic}{
  short = DIC,
  long  = Ductal Inflating Carcinoma
}

\DeclareAcronym{ilc}{
  short = ILC,
  long  = Inflating Lobular Carcinoma
}

\DeclareAcronym{fpr}{
  short = FPR,
  long  = False Positive Ratio
}

\DeclareAcronym{fnr}{
  short = FNR,
  long  = False Negative Ratio
}

\DeclareAcronym{fp}{
  short = FP,
  long  = False Positive
}


      % contains the acronims

%% Select inputing only one part of the document
%\includeonly{content/intro/intro}   % the file wihtout .tex
%\includeonly{content/other/other_content}

\begin{document}
\maketitle

\begin{abstract}
\acresetall  % reset the acronyms from the title (if any)
  Breast cancer is the second most common cancer and the leading cause of cancer death among women.
  Medical imaging has become an indispensable tool for its diagnosis and follow up.
  During the last decade, the medical community has promoted to incorporate \ac{us} screening as part of the standard routine.
  The main reason for using \ac{us} imaging is its capability to differentiate benign from malignant masses, when compared to other imaging techniques.
  The increasing usage of \ac{us} imaging encourages the development of \ac{cad} systems applied to \ac{bus} images.
  However accurate delineations of the lesions and structures of the breast are essential for \ac{cad} systems in order to extract information needed to perform diagnosis.

  This article proposes a highly modular and flexible framework for segmenting lesions and tissues present in \ac{bus} images.
  The proposal takes advantage of optimization strategies using super-pixels and high-level descriptors, which are analogous to the visual cues used by radiologists.
  Qualitative and quantitative results are provided stating a performance within the range of the state-of-the-art.
\end{abstract}

\keywords{Breast Ultra-Sound, BI-RADS lexicon, Optimization based Segmentation, Machine-Learning based Segmentation, Graph-Cuts}

%% Incldue the content without .tex extension
\acresetall  % reset the acronyms from the abstract
\include*{content/intro/intro}          % the file wihtout .tex
\include*{content/method/method}
%\include*{content/features/features}
\include*{content/results/results}

\section{Conclusions}
This work presents a segmentation strategy to delineate lesions in \ac{bus} images using an optimization framework that takes advantage of all the facilities available when using \ac{ml} techniques.
Despite the limitation that the final segmentation is subject to the super-pixels' boundaries, the \ac{aov} results here reported are similar to those reported by other methodologies in the literature.
A higher \ac{aov} result can be achieved by deforming the delineation resulting from the proposed framework using this to initialize a second post-processing step based on \ac{acm}. In this manner the contour constrains could be applied to achieve a more natural delineation.

%\acknowledgments     %>>>> equivalent to \section*{ACKNOWLEDGMENTS}
%
%This unnumbered section is used to identify those who have aided the authors in understanding or accomplishing the work presented and to acknowledge sources of funding.

%%%%%%%%%%%%%%%%%%%%%%%%%%%%%%%%%%%%%%%%%%%%%%%%%%%%%%%%%%%%%
%%%%% References %%%%%


\bibliography{./content/lit_review}   %>>>> bibliography data in report.bib
\bibliographystyle{spiebib}   %>>>> makes bibtex use spiebib.bst

\end{document}

