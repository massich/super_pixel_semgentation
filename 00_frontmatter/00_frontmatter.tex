\title{Style template and guidelines for SPIE Proceedings} 

%>>>> The author is responsible for formatting the 
%  author list and their institutions.  Use  \skiplinehalf 
%  to separate author list from addresses and between each address.
%  The correspondence between each author and his/her address
%  can be indicated with a superscript in italics, 
%  which is easily obtained with \supit{}.

\author{Anna A. Author1\supit{a} and Barry B. Author2\supit{b}
\skiplinehalf
\supit{a}Affiliation1, Address, City, Country; \\
\supit{b}Affiliation2, Address, City, Country
}

%>>>> Further information about the authors, other than their 
%  institution and addresses, should be included as a footnote, 
%  which is facilitated by the \authorinfo{} command.

\authorinfo{Further author information: (Send correspondence to A.A.A.)\\A.A.A.: E-mail: aaa@tbk2.edu, Telephone: 1 505 123 1234\\  B.B.A.: E-mail: bba@cmp.com, Telephone: +33 (0)1 98 76 54 32}
%%>>>> when using amstex, you need to use @@ instead of @

\maketitle 

\begin{abstract}
This document shows the desired format and appearance of a manuscript prepared for the Proceedings of the SPIE.  It contains general formatting instructions and hints about how to use LaTeX.  The LaTeX source file that produced this document, {\tt article.tex} (Version 3.2), provides a template, which can be used in conjunction with {\tt spie.cls}.  
\end{abstract}

\keywords{Manuscript format, template, SPIE Proceedings, LaTeX}

%%% Local Variables: 
%%% mode: latex
%%% TeX-master: "../master.tex"
%%% End: 