% include the figures path relative to the master file
\graphicspath{ {./content/intro/figures/} }

\section{Introduction}
\label{sec:intro}  % \label{} allows reference to this section


Breast cancer is the second most common cancer (1.4 million cases per year, 10.9\% of  diagnosed cancers) after lung cancer, followed by colorectal, stomach, prostate and liver cancers~\cite{Ferlay2010}.
In terms of mortality, breast cancer is the fifth most common cause of cancer death.
However, it is ranked as the leading cause of cancer death among females in both western countries and economically developing countries~\cite{cancerStatistics2011}.

Medical imaging plays an important role in breast cancer mortality reduction, contributing to its early detection through screening for diagnosis, image-guided biopsy, treatment follow-up and suchlike procedures~\cite{smith2003american}.
Although \ac{dm} remains the reference imaging modality for breast cancer screening, \ac{us} imaging has proven to be a successful adjunct image modality~\cite{smith2003american,berg2004diagnostic}.
The main advantage of \ac{us} imaging, opposed to other image modalities, lies on the discriminative power \ac{us} offers for visually differentiate benign from malignant solid lesions~\cite{Stavros:1995p12392}.
In this manner, \ac{us} screening contributes to reduce the amount of unnecessary biopsies~\cite{ciatto1994contribution}, which is estimated to be between $65\sim85\%$ of the prescribed biopsies~\cite{yuan2010multimodality}, in favour of a less traumatic short-term screening follow-up using \ac{bus}~\cite{gordon1995malignant}
For all these reasons, there is a growing interest in the medical community to incorporate \ac{us} screening as part of the standard procedure~\cite{biradsus}, which encourages the development of \ac{cad} systems using \ac{us} to be applied to breast cancer diagnosis.

The \ac{acr}, in order to provide a common ground for radiologists when assessing \ac{bus} images, compiled and proposed the \ac{birads} lexicon for \ac{bus} images~\cite{biradsus}.
A lexicon is a standardized set of markers to describe the visual cues found in \ac{bus} images that are recommended to be analysed when performing image based diagnosis.
This lexicon, proposed by the \ac{acr}, can be found in this document at \cref{sec:method:dterm:feat}, where visual cues of \ac{bus} images and breast structures are discussed to define feature descriptors.
While visual cues are discussed further in this document, it is worth to mention here that the \ac{us} \ac{birads} lexicon is designed to be used by expert radiologists to characterize the lesions and produce a diagnosis based on the lexicon description of the lesions.
This implies that radiologists, during the visual assessment of the images,  locate the lesions and determine their extension prior to utilize the lexicon. Obviously, this is an intrinsic the process carried out by the trained radiologists when visually reading the images and there is no need for explicit delineation of the lesions.
However, developing accurate segmentation methodologies breast lesions and structures is crucial for developing \ac{cad} systems that can take advantage of the already existing tools for characterizing the lesions.

%Regardless of the clinical utility of the \ac{us} images, such image modality suffers from different inconveniences due to strong noise natural of \ac{us} imaging  and the presence of strong \ac{us} artifacts, both degrading the overall image quality~\cite{Ensminger:2008p6920} which compromise the performance of the radiologists.
%Radiologists infer health state of the patients based on visual inspection of images which by means of some screening technique (e.g.~\ac{us}) depict physical properties of the screened body.
%The radiologic diagnosis error rates are similar to those found in any other tasks requiring human visual inspection, and such errors, are subject to the quality of the images and the ability of the reader to interpret the physical properties depicted on them\cite{manning2005perception}.
%
%Therefore the interest from the medical imaging community, also for the specific case of breast lesion assessment using \ac{us} data, in developing \ac{cad} systems that provide better instrumentation to improve image interpretation, and consequently achieve better diagnosis.

This article proposes a highly modular and flexible framework for segmenting lesions and tissues present in \ac{bus} images.
The proposal takes advantage of an energy-based strategy to perform segmentations based on discrete optimizations using super-pixels and a set of novel features analogous to the elements encoded by the \ac{us} \ac{birads} lexicon~\cite{biradsus}.

%%% Local Variables: 
%%% mode: latex
%%% TeX-master: "../../master.tex"
%%% End: \section{introduction}
