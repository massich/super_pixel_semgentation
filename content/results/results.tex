% include the figures path relative to the master file
\graphicspath{ {./content/results/figures/} }

\section{Results} 
A lack of public data or benchmark to perform fair comparison between methodologies is a recurrent problem in medical imaging.
\ac{bus} imaging is not an exception \cite{Cheng:2009p10580}. 
Therefore, our framework can only be compared for the lesions segmentation case, and only against the results reported in the bibliography.

\Cref{fig:surveyResults} compares our segmentation strategy against the state-of-the-art methodologies assuming the following limitations:
(a) the other methodologies performances have been collected from the literature; (b) since all the segmentation results are reported using different metrics, those have been translated to \ac{aov} as a common evaluation metric; and (c) the evaluation datasets differ regarding the patient identities but also the number of patients included in these datasets.

Each radius (a .. p) represents a methodology from the literature.
Those methodologies have been grouped in terms of \ac{ml}, \ac{acm}, other methodologies, and combinations of those three classes.
The figure depicts the size of the used dataset for evaluation and also the \ac{aov} reported. 
Highlighted in blue there is also represented an experiment conduced by Pons et al.~\cite{gerard2013} where 50 \ac{bus} images with a single lesion were all delineated by 5 experts in order to study inter- and intra-observer variability of \ac{gt} annotation. 
The experiment reported an \ac{aov} rate between $0.8$ and $0.852$ for the 6 actors, when counting the original \ac{gt} accompanying the images.

Our segmentation results are represented as a black circle showing that those are within the state-of-the-art. 
A more meticulous analysis of the results will be presented in the complete version of the manuscript.

\begin{figure}[h]
  \centering
  \begin{tikzpicture}[scale=.58]

\definecolor{autoGuided}{rgb}{ 0.3765    0.7294    0.9412}
\newcommand{\autoGuidedColor}{(light-Blue)}
\definecolor{fullyAuto}{rgb}{ 0.0941    0.3843    0.6627}
\newcommand{\fullyAutoColor}{(dark-blue)}
\definecolor{semiAuto}{rgb}{ 0.0784    0.5059    0.1686}
\newcommand{\semiAutoColor}{(light-green)}
\definecolor{fullyGuided}{rgb}{ 0.4275    0.6902    0.3176}
\newcommand{\fullyGuidedColor}{(dark-green)}
\definecolor{colorTheme}{RGB}{51,41,178}

\def\labels{	{\color{semiAuto}[68]},%\cite{AlemanFlores:2007p14310}},
						{\color{semiAuto}  [70]},%\cite{Gao:2012p14336}},
						{\color{fullyAuto}[85]},%\cite{Liu:2010p14328}},
						{\color{semiAuto}[69]},%\cite{Cui:2009p14325}},					
						{\color{autoGuided}[75]},%\cite{Huang:2007p6100}},
						{\color{fullyAuto}[80]},%\cite{Huang:2005p11636}},
						{\color{semiAuto}[64]},%\cite{Gomez:2010p14339}},
						{\color{semiAuto}[58]},%\cite{Horsch:2001p6028}},
						{\color{fullyAuto}[86]},%\cite{Yeh:2009p11985}},
						{\color{autoGuided}[79]},%\cite{Shan:2012p14347}},
						{\color{autoGuided}[61]},%\cite{massich2010lesion}},
						{\color{semiAuto}[66]},%\cite{Xiao:2002p5639,gerard2013}},
						{\color{autoGuided}[76]},%\cite{Zhang:2010p14317}},
						{\color{fullyAuto}[84]},%\cite{hao2012combining}},
						{\color{autoGuided}[60]},%\cite{Madabhushi:2003p6036}},
						{\color{fullyAuto}[81]}}%\cite{Huang:2012p14313}}}
						
\def\reward{88.3,86.3,88.1,74.5,77.6,78.6,85.0,73.0,73.3,83.1,64.0,54.9,84.0,75.0,62.0,85.2}
\def\dbSize{32,20,76,488,118,20,50,400,6,120,25,352,347,480,42,20}
\def\dbClass{1,1,2,3,2,1,2,3,1,2,1,3,3,3,1,1}		
\def\cZoom{3} 
\def\percentageLabelAngle{90}
\def\nbeams{16}
\pgfmathsetmacro\beamAngle{(360/\nbeams)}
\pgfmathsetmacro\halfAngle{(180/\nbeams)}
%\def\globalRotation{10}
\pgfmathsetmacro\globalRotation{\halfAngle}

% draw manual AOV results
\filldraw[blue!15!white,even odd rule] (0,0) circle [radius={\cZoom*.852}] (0,0) circle [radius={\cZoom*.8}];
\draw[thin,color=blue!50!white,dashed] (0,0) circle [radius={\cZoom*.852}] (0,0) circle [radius={\cZoom*.8}];

%\foreach \x in {.125,.25, ...,1} { \draw[thin]  (0,0) circle [radius={2*\x}]; }
% draw the radiants with the reference label
\foreach \n  [count=\ni] in \labels
{
\pgfmathsetmacro\cAngle{{(\ni*(360/\nbeams))+\globalRotation}}
\draw [thin] (0,0) -- (\cAngle:{\cZoom*1}) ;
\draw	(\cAngle:{\cZoom*1.1})  node[fill=white, inner sep=0pt] {{\tiny \textbf  \n}}; %referencies
}

% draw the % rings 
\foreach \x in {12.5,25, ...,100} 
\draw [thin,color=gray!50] (0,0) circle [radius={\cZoom*\x/100}];

\foreach \x in {50,75,100}
{ 
     \draw [thin,color=black!50] (0,0) circle [radius={\cZoom/100*\x}];
     \foreach \a in {0, 180} \draw ({\percentageLabelAngle+\a}:{\cZoom*0.01*\x}) node  [inner sep=0pt,outer sep=0pt,fill=white,font=\fontsize{5}{5}\selectfont]{$\x$};
}


% draw the path of the percentages
\def\aux{{\reward}}
\pgfmathsetmacro\origin{\aux[\nbeams-1]} 
\draw [blue, thick] (\globalRotation:{\cZoom*\origin/100}) \foreach \n  [count=\ni] in \reward { -- ({(\ni*(360/\nbeams))+\globalRotation}:{\cZoom*\n/100}) } ;

% label all the percentags
\foreach \n [count=\ni] in \dbSize 
{
	\pgfmathsetmacro\cAngle{{(\ni*(360/\nbeams))+\globalRotation}}
	\pgfmathsetmacro\nreward{\aux[\ni-1]}
	\draw (\cAngle:{\cZoom*1.4}) node[align=center] {{\color{blue}\nreward $\%$} \\ {\color{red}\n} };
} ;

% draw the database rose
\def\dbScale{\9}
\foreach \n [count=\ni] in \dbClass
\filldraw[fill=red!20!white, draw=red!50!black]
(0,0) -- ({\ni*(360/\nbeams)-\halfAngle+\globalRotation}:{\cZoom*\n/9}) arc ({\ni*(360/\nbeams)-\halfAngle+\globalRotation}:{\ni*(360/\nbeams)+\halfAngle+\globalRotation}:{\cZoom*\n/9}) -- cycle;
\foreach \x in {1,2,3}
\draw [thin,color=red!50!black,dashed] (0,0) circle [radius={\cZoom*\x/9}];

%% draw the domain of each class 
  \def\puta{	3/0/{ACM},
  			3/3/{ACM+Other},
  			3/6/{Other}}
\def\putaa{  	2/9/{Other+ML},
  			3/11/{ML},
  			2/14/{ML+ACM}}

\foreach \numElm/\contadorQueNoSeCalcular/\name [count=\ni] in \puta
 {

 	\pgfmathsetmacro\initialAngle{(\contadorQueNoSeCalcular*\beamAngle)+\halfAngle+\globalRotation}
 	\pgfmathsetmacro\finalAngle  {((\numElm+\contadorQueNoSeCalcular)*\beamAngle)+\halfAngle+\globalRotation}
	\pgfmathsetmacro\l  {\cZoom*1.5+.3pt}
	\draw (\initialAngle:{\cZoom*1.6}) -- (\initialAngle:{\cZoom*1.1});
	\draw [ |<->|,>=latex] (\initialAngle:\l) arc (\initialAngle:\finalAngle:\l) ;    									 
	\pgfmathsetmacro\r  {\cZoom*1.5+.45pt}
    	{\draw [decoration={text along path,  text={\name},text align={center}},decorate] (\finalAngle:\r) arc (\finalAngle:\initialAngle:\r);}
  }
  
   \foreach \numElm/\contadorQueNoSeCalcular/\name [count=\ni] in \putaa
 {

 	\pgfmathsetmacro\initialAngle{(\contadorQueNoSeCalcular*\beamAngle)+\halfAngle+\globalRotation}
 	\pgfmathsetmacro\finalAngle  {((\numElm+\contadorQueNoSeCalcular)*\beamAngle)+\halfAngle+\globalRotation}
	\pgfmathsetmacro\l  {\cZoom*1.5+.3pt}
	\draw (\initialAngle:{\cZoom*1.6}) -- (\initialAngle:{\cZoom*1.1});
	\draw [ |<->|,>=latex] (\initialAngle:\l) arc (\initialAngle:\finalAngle:\l) ;    									 
	\pgfmathsetmacro\r  {\cZoom*1.5+.7pt}
    	{\draw [decoration={text along path, text={\name},text align={center}},decorate] (\initialAngle:\r) arc (\initialAngle:\finalAngle:\r);}    			 
  }
  
    \draw [thick,color=black] (0,0) circle [radius={\cZoom*.63}];
  
    \node [anchor=north west] at (\cZoom*1.8,\cZoom*1.5){
	\begin{tikzpicture}
  \begin{customlegend}[legend entries={Dataset size categorization,Datset size,AOV results, Manual delineation AOV, AOV percentage,$62.3\%$ AOV}]
    \addlegendimage{red,fill=red!20!white, draw=red!60!gray, ybar, ybar legend}
    \addlegendimage{number in legend=XX,red}
    \addlegendimage{red,draw=blue,thick,sharp plot}
    \addlegendimage{red,fill=blue!15!white,draw=blue,dashed,area legend}
    \addlegendimage{number in legend=XX.X\%,blue}
    \addlegendimage{red,draw=black,thick,sharp plot}

    \end{customlegend}
\end{tikzpicture}}; 

\end{tikzpicture}
%\caption{{\tiny Circular representation of the reported results grouped by methodology class. From inside to outside: used dataset categorization, AOV, work ID, numeric data and methodology category label. The work ID is colored as follows: semi-automatic {\color{semiAuto}\semiAutoColor}, auto-guided{\color{autoGuided}\autoGuidedColor}, and fully automatic{\color{fullyAuto}\fullyAutoColor}.}} 
%\end{figure}
%\

  \caption{Quantitative AOV results}
  \label{fig:surveyResults}
\end{figure}


%%% Local Variables: 
%%% mode: latex
%%% TeX-master: "../../master.tex"
