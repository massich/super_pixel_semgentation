% include the figures path relative to the master file
\graphicspath{ {./content/results/figures/} }

\section{Method evaluation and comparison} 
A 16 \ac{bus} images dataset with accompanying multi-label \ac{gt} delineating all the structures present in the images has been used to evaluate the proposed methodology for lesion segmentation application.
Every image in the used dataset, presents a single lesion with variable extension. 
The size of the lesions ranges from under one hundreth to over one fifth of the image.
The dataset composed of cysts, \acp{fa}, \acp{dic} and \acp{ilc}.

\Cref{fig:results} shows qualitative results, results whereas the quantitative results from the best configuration are reported as a table in \cref{fig:surveyResults:method}~\cite{massich2013phd}.
Notice that for the proposed framework, the performance in terms of \ac{aov} is limitted by the capacity of the supper-pixels to snap de desired boundary.
\Cref{fig:results:perfect} shows how the delination resulting from a propoer labeling of the super-pixels differs from the \ac{gt}.
Despite the fact that a \ac{fpr} of $0.4$ seems significant, based on our experiments most of the images produce no \ac{fp} lessions.
However images producing \ac{fp} liesions, are likely to produce more than a single \ac{fp} (see \cref{fig:results}c-e).
Therefore its presence in the results~\cite{massich2013phd}.
The amount of \ac{fp} lesions can be trimmed by applying a higher cost in the pairwise therm (compare \cref{fig:resuts:smallPWterm} and \cref{fig:results:bigPWterm}.
Nevertheless, a severe incresing of the pairwise cost also incrases the \ac{fnr} since some lesions are missed due to over-smoothing~\cite{massich2013phd}.
The situation of having a larger \ac{fnr} is less desirable than reducing the \ac{fpr}. 
The \ac{fnr} reported in \cref{fig:surveyResults:method} is caused by an image within the dataset that its lesion is fully contained in a single super-pixel and still around $20\%$ of this super-pixel's area is healty tissue. 

\begin{figure}[h]
  \centering
    \begin{subfigure}[b]{0.4\textwidth}
      \begin{subfigure}[b]{0.48\textwidth}
          \includegraphics[width=\textwidth]{goodQSorigin}
          \caption{xx}
      \end{subfigure}
      \hfill
      \begin{subfigure}[b]{0.48\textwidth}
          \includegraphics[width=\textwidth]{goodQSseg}
          \caption{xx}
          \label{fig:results:perfect}
      \end{subfigure}
    \end{subfigure}
    \hfill
    \begin{subfigure}[b]{0.45\textwidth}
      \begin{subfigure}[b]{0.65\textwidth}
          \includegraphics[width=\textwidth]{fporigin}
          \caption{xx}
      \end{subfigure}
      \hfill
      \begin{subfigure}[b]{0.28\textwidth}
        \begin{subfigure}[b]{\textwidth}
            \includegraphics[width=\textwidth]{fpnohom}
            \caption{xx}
            \label{fig:resuts:smallPWterm}
        \end{subfigure}
        \begin{subfigure}[b]{\textwidth}
            \includegraphics[width=\textwidth]{fpHom}
            \caption{{\small xx}}
            \label{fig:results:bigPWterm}
        \end{subfigure}
      \end{subfigure}
    \end{subfigure}
  \caption{Qualitative results}
  \label{fig:results}
\end{figure}

Due to the lack of publicly available data (and code) there is no manner to perform a methodology comparison further than compiling the results reported in the literature.
The table in \Cref{fig:surveyResults:survey} complies the evaluation reported by the authors of the most relevant methodologies found in the literature~\cite{massich2013phd}.
Details about the methodologies proposed in the literature can also be found in the aforesaid table.
Specifically it is detailed the category of technique been used for detecting the lesions, segmenting it and post-process the delineations (if any).
The studied categories are: \ac{ml}, \ac{acm} and others.
The iconography used in \cref{fig:surveyResults:survey} also illustrates if those stages are treated as independent and connected in a daisy-chain fashion, or otherwise the stages are addressed in an atomic manner.

\Cref{fig:surveyResults:comparison} renders the information present in \cref{fig:surveyResults:survey} and \cref{fig:surveyResults:method} in a visual manner compare all the results at once.
The methodologies arranged in a radial fashion and grouped by its most representative technology category. 
In red it can be found a small, medium and large categorization of the dataset reported for testing.
The concentric circles represent \ac{aov}. 
The blue line correspond to the \ac{aov} results reported in the literature, whereas the black line indicates the \ac{aov} our framework scored in our testing.
An extra element is also represented in \cref{fig:surveyResults:comparison} as blue swatch delimited by two blue dashed lines.
The boundaries of this swatch correspond to performance of expert radiologists in terms of \ac{aov} based on an inter- and intra-observer experiment carried out by Pons et al.~\cite{gerard2013}\footnote{The dataset used for testing the framework here proposed corresponds to the subset of images used by Pons et al.~\cite{gerard2013} that have accompanying multi-labelled \ac{gt}.}.

When comparing the results it is clear the inconvenience of unexciting public data, since several of the results outperform the manual delineations studied in~\cite{gerard2013}.
It can also be seen that the category tested in larger datasets is \ac{ml}, whereas \ac{acm} lead to better segmentations since the lesion boundary is easier to model in \ac{acm} compared to \ac{ml} based techniques.


\begin{figure}[h]
  \begin{subfigure}[b]{\textwidth}
    {\tiny 
\newcommand{\myCoord}[1]{
  \tikz[remember picture]{\coordinate[remember picture] (#1) at (0,0);
    \node at (#1) {x};
  }
}

\begin{tabular}{lcccccccccccccccc}
  Method Id:
              &a\cite{Liu:2010p14328}
              &b\cite{Gao:2012p14336}
              &c\cite{AlemanFlores:2007p14310}
              &d\cite{Huang:2012p14313}
              &e\cite{Madabhushi:2003p6036}
              &f\cite{hao2012combining}
              &g\cite{Zhang:2010p14317}
              &h\cite{Xiao:2002p5639}
              &i\cite{massich2010lesion}
              &j\cite{Shan:2012p14347}
              &k\cite{Yeh:2009p11985}
              &l\cite{Horsch:2001p6028}
              &m\cite{Gomez:2010p14339}
              &n\cite{Huang:2005p11636}
              &o\cite{Huang:2007p6100}
              &p\cite{Cui:2009p14325}\\
  \hline
  Dataset size:     & 76   & 20   & 32   & 20   & 42   & 480  & 347    & 352  & 25
                    & 120  & 6    & 400  & 50   & 20   & 118  & 488 \\

  \ac{aov} (in \%): & 88.1 & 86.3 & 88.3 & 85.2 & 62.0 & 75.0 & 84.0   & 54.9 & 64.0
                    & 83.1 & 73.3 & 73.0 & 85.0 & 78.6 & 77.6 & 74.5\\

  \hline
  techonlogy used for:  &\\
  \quad detection       & \myCoord{Adetect} & \myCoord{Bdetect} & \myCoord{Cdetect} & \myCoord{Ddetect} & \myCoord{Edetect} 
                        & \myCoord{Fdetect} & \myCoord{Gdetect} & \myCoord{Hdetect} & \myCoord{Idetect} & \myCoord{Jdetect}
                        & \myCoord{Kdetect} & \myCoord{Ldetect} & \myCoord{Mdetect} & \myCoord{Ndetect} & \myCoord{Odetect}
                        & \myCoord{Pdetect}\\

  \quad segmetnation    & \myCoord{Aseg} & \myCoord{Bseg} & \myCoord{Cseg} & \myCoord{Dseg} & \myCoord{Eseg} 
                        & \myCoord{Fseg} & \myCoord{Gseg} & \myCoord{Hseg} & \myCoord{Iseg} & \myCoord{Jseg}
                        & \myCoord{Kseg} & \myCoord{Lseg} & \myCoord{Mseg} & \myCoord{Nseg} & \myCoord{Oseg}
                        & \myCoord{Pseg}\\

  \quad post-processing & \myCoord{App} & \myCoord{Bpp} & \myCoord{Cpp} & \myCoord{Dpp} & \myCoord{Epp} 
                        & \myCoord{Fpp} & \myCoord{Gpp} & \myCoord{Hpp} & \myCoord{Ipp} & \myCoord{Jpp}
                        & \myCoord{Kpp} & \myCoord{Lpp} & \myCoord{Mpp} & \myCoord{Npp} & \myCoord{Opp}
                        & \myCoord{Ppp}\\
\end{tabular}

\begin{tikzpicture}[remember picture]
  \node at (Adetect) {z};
  % \foreach \x in {Adetect, Bdetect, Aseg, Bpp}
  % \node at (\x) {y};
\end{tikzpicture}
}
    \caption{\ac{bus} images lesion segmentation strategies compiled from the bulk of the literature: reported quantitative results and methodology highlights.}
    \label{fig:surveyResults:survey}
  \end{subfigure}
  \begin{subfigure}[b]{\textwidth}
    \begin{subfigure}[b]{0.2\textwidth}
      \begin{tabular}{lc}
        \ac{aov}& .623 \\
        \ac{fpr}& .4 \\
        \ac{fnr}& .008 \\
      \end{tabular}
      \caption{xx}
      \label{fig:surveyResults:method}
    \end{subfigure}
    \begin{subfigure}[b]{0.8\textwidth}
      {\tiny \begin{tikzpicture}[scale=.58]

\definecolor{autoGuided}{rgb}{ 0.3765    0.7294    0.9412}
\newcommand{\autoGuidedColor}{(light-Blue)}
\definecolor{fullyAuto}{rgb}{ 0.0941    0.3843    0.6627}
\newcommand{\fullyAutoColor}{(dark-blue)}
\definecolor{semiAuto}{rgb}{ 0.0784    0.5059    0.1686}
\newcommand{\semiAutoColor}{(light-green)}
\definecolor{fullyGuided}{rgb}{ 0.4275    0.6902    0.3176}
\newcommand{\fullyGuidedColor}{(dark-green)}
\definecolor{colorTheme}{RGB}{51,41,178}

\def\labels{	{\color{semiAuto}[68]},%\cite{AlemanFlores:2007p14310}},
						{\color{semiAuto}  [70]},%\cite{Gao:2012p14336}},
						{\color{fullyAuto}[85]},%\cite{Liu:2010p14328}},
						{\color{semiAuto}[69]},%\cite{Cui:2009p14325}},					
						{\color{autoGuided}[75]},%\cite{Huang:2007p6100}},
						{\color{fullyAuto}[80]},%\cite{Huang:2005p11636}},
						{\color{semiAuto}[64]},%\cite{Gomez:2010p14339}},
						{\color{semiAuto}[58]},%\cite{Horsch:2001p6028}},
						{\color{fullyAuto}[86]},%\cite{Yeh:2009p11985}},
						{\color{autoGuided}[79]},%\cite{Shan:2012p14347}},
						{\color{autoGuided}[61]},%\cite{massich2010lesion}},
						{\color{semiAuto}[66]},%\cite{Xiao:2002p5639,gerard2013}},
						{\color{autoGuided}[76]},%\cite{Zhang:2010p14317}},
						{\color{fullyAuto}[84]},%\cite{hao2012combining}},
						{\color{autoGuided}[60]},%\cite{Madabhushi:2003p6036}},
						{\color{fullyAuto}[81]}}%\cite{Huang:2012p14313}}}
						
\def\reward{88.3,86.3,88.1,74.5,77.6,78.6,85.0,73.0,73.3,83.1,64.0,54.9,84.0,75.0,62.0,85.2}
\def\dbSize{32,20,76,488,118,20,50,400,6,120,25,352,347,480,42,20}
\def\dbClass{1,1,2,3,2,1,2,3,1,2,1,3,3,3,1,1}		
\def\cZoom{3} 
\def\percentageLabelAngle{90}
\def\nbeams{16}
\pgfmathsetmacro\beamAngle{(360/\nbeams)}
\pgfmathsetmacro\halfAngle{(180/\nbeams)}
%\def\globalRotation{10}
\pgfmathsetmacro\globalRotation{\halfAngle}

% draw manual AOV results
\filldraw[blue!15!white,even odd rule] (0,0) circle [radius={\cZoom*.852}] (0,0) circle [radius={\cZoom*.8}];
\draw[thin,color=blue!50!white,dashed] (0,0) circle [radius={\cZoom*.852}] (0,0) circle [radius={\cZoom*.8}];

%\foreach \x in {.125,.25, ...,1} { \draw[thin]  (0,0) circle [radius={2*\x}]; }
% draw the radiants with the reference label
\foreach \n  [count=\ni] in \labels
{
\pgfmathsetmacro\cAngle{{(\ni*(360/\nbeams))+\globalRotation}}
\draw [thin] (0,0) -- (\cAngle:{\cZoom*1}) ;
\draw	(\cAngle:{\cZoom*1.1})  node[fill=white, inner sep=0pt] {{\tiny \textbf  \n}}; %referencies
}

% draw the % rings 
\foreach \x in {12.5,25, ...,100} 
\draw [thin,color=gray!50] (0,0) circle [radius={\cZoom*\x/100}];

\foreach \x in {50,75,100}
{ 
     \draw [thin,color=black!50] (0,0) circle [radius={\cZoom/100*\x}];
     \foreach \a in {0, 180} \draw ({\percentageLabelAngle+\a}:{\cZoom*0.01*\x}) node  [inner sep=0pt,outer sep=0pt,fill=white,font=\fontsize{5}{5}\selectfont]{$\x$};
}


% draw the path of the percentages
\def\aux{{\reward}}
\pgfmathsetmacro\origin{\aux[\nbeams-1]} 
\draw [blue, thick] (\globalRotation:{\cZoom*\origin/100}) \foreach \n  [count=\ni] in \reward { -- ({(\ni*(360/\nbeams))+\globalRotation}:{\cZoom*\n/100}) } ;

% label all the percentags
\foreach \n [count=\ni] in \dbSize 
{
	\pgfmathsetmacro\cAngle{{(\ni*(360/\nbeams))+\globalRotation}}
	\pgfmathsetmacro\nreward{\aux[\ni-1]}
	\draw (\cAngle:{\cZoom*1.4}) node[align=center] {{\color{blue}\nreward $\%$} \\ {\color{red}\n} };
} ;

% draw the database rose
\def\dbScale{\9}
\foreach \n [count=\ni] in \dbClass
\filldraw[fill=red!20!white, draw=red!50!black]
(0,0) -- ({\ni*(360/\nbeams)-\halfAngle+\globalRotation}:{\cZoom*\n/9}) arc ({\ni*(360/\nbeams)-\halfAngle+\globalRotation}:{\ni*(360/\nbeams)+\halfAngle+\globalRotation}:{\cZoom*\n/9}) -- cycle;
\foreach \x in {1,2,3}
\draw [thin,color=red!50!black,dashed] (0,0) circle [radius={\cZoom*\x/9}];

%% draw the domain of each class 
  \def\puta{	3/0/{ACM},
  			3/3/{ACM+Other},
  			3/6/{Other}}
\def\putaa{  	2/9/{Other+ML},
  			3/11/{ML},
  			2/14/{ML+ACM}}

\foreach \numElm/\contadorQueNoSeCalcular/\name [count=\ni] in \puta
 {

 	\pgfmathsetmacro\initialAngle{(\contadorQueNoSeCalcular*\beamAngle)+\halfAngle+\globalRotation}
 	\pgfmathsetmacro\finalAngle  {((\numElm+\contadorQueNoSeCalcular)*\beamAngle)+\halfAngle+\globalRotation}
	\pgfmathsetmacro\l  {\cZoom*1.5+.3pt}
	\draw (\initialAngle:{\cZoom*1.6}) -- (\initialAngle:{\cZoom*1.1});
	\draw [ |<->|,>=latex] (\initialAngle:\l) arc (\initialAngle:\finalAngle:\l) ;    									 
	\pgfmathsetmacro\r  {\cZoom*1.5+.45pt}
    	{\draw [decoration={text along path,  text={\name},text align={center}},decorate] (\finalAngle:\r) arc (\finalAngle:\initialAngle:\r);}
  }
  
   \foreach \numElm/\contadorQueNoSeCalcular/\name [count=\ni] in \putaa
 {

 	\pgfmathsetmacro\initialAngle{(\contadorQueNoSeCalcular*\beamAngle)+\halfAngle+\globalRotation}
 	\pgfmathsetmacro\finalAngle  {((\numElm+\contadorQueNoSeCalcular)*\beamAngle)+\halfAngle+\globalRotation}
	\pgfmathsetmacro\l  {\cZoom*1.5+.3pt}
	\draw (\initialAngle:{\cZoom*1.6}) -- (\initialAngle:{\cZoom*1.1});
	\draw [ |<->|,>=latex] (\initialAngle:\l) arc (\initialAngle:\finalAngle:\l) ;    									 
	\pgfmathsetmacro\r  {\cZoom*1.5+.7pt}
    	{\draw [decoration={text along path, text={\name},text align={center}},decorate] (\initialAngle:\r) arc (\initialAngle:\finalAngle:\r);}    			 
  }
  
    \draw [thick,color=black] (0,0) circle [radius={\cZoom*.63}];
  
    \node [anchor=north west] at (\cZoom*1.8,\cZoom*1.5){
	\begin{tikzpicture}
  \begin{customlegend}[legend entries={Dataset size categorization,Datset size,AOV results, Manual delineation AOV, AOV percentage,$62.3\%$ AOV}]
    \addlegendimage{red,fill=red!20!white, draw=red!60!gray, ybar, ybar legend}
    \addlegendimage{number in legend=XX,red}
    \addlegendimage{red,draw=blue,thick,sharp plot}
    \addlegendimage{red,fill=blue!15!white,draw=blue,dashed,area legend}
    \addlegendimage{number in legend=XX.X\%,blue}
    \addlegendimage{red,draw=black,thick,sharp plot}

    \end{customlegend}
\end{tikzpicture}}; 

\end{tikzpicture}
%\caption{{\tiny Circular representation of the reported results grouped by methodology class. From inside to outside: used dataset categorization, AOV, work ID, numeric data and methodology category label. The work ID is colored as follows: semi-automatic {\color{semiAuto}\semiAutoColor}, auto-guided{\color{autoGuided}\autoGuidedColor}, and fully automatic{\color{fullyAuto}\fullyAutoColor}.}} 
%\end{figure}
%\
 }
      \caption{{\small xx}}
      \label{fig:surveyResults:comparison}
    \end{subfigure}
  \end{subfigure}
  \hfill
  \caption{Quantitative results compilation and comparison}
  \label{fig:surveyResults}
\end{figure}


%%% Local Variables: 
%%% mode: latex
%%% TeX-master: "../../master.tex"
