% include the figures path relative to the master file
\graphicspath{ {./content/method/figures/} }

\section{Segmentation methodology description} 

Optimization methodologies offer a standardized manner to approach segmentation by minimizing an application-driven cost function~\cite{cremers2007review}.
The segmentation strategy here adopted is formulated as a \emph{metric labelling} problem. 
The image is conceived as a discrete set of elements $\mathcal{S}$ that need to be labelled using a label $l$ from a labelling set $\mathcal{L}$ (i.e.\, $l \in \text{\{lesion, lesion\}}$ or $l \in \text{\{lungs, fat, parenchyma, .. ., lesion\}}$).
Let $\mathcal{W}$ be all the possible labelling configurations of the set $\mathcal{S}$ given $\mathcal{L}$, and let $U(\cdot)$ be a cost function encoding how good is a labelling configuration $\omega \in \mathcal{W}$ based on the appearance of the elements in $\mathcal{S}$, their relation and some designing constrains.
Then the desired segmentation $\hat{\omega}$, corresponds to the labelling configuration minimizing this cost function, $\displaystyle \hat{\omega} = \arg \min_{\substack{\omega}} \,U(\omega)$. 

For further interpretation, \cref{eq:labelingEq} shows the cost details; 
\cref{fig:method} illustrates a generic version of the segmentation strategy here proposed; and, \cref{fig:methodTerms} offers an interpretation of the terms found in \cref{eq:labelingEq,fig:method} applied to breast lesion segmentation in \ac{us} images.

\begin{equation}
  U(\omega) = \sum_{s\in S} D_s(\omega_s) + \sum_{s}\sum_{r \in \mathcal{N}_{s}} V_{s,r}(\omega_s,\omega_r)
  \label{eq:labelingEq}
\end{equation}

The $U(\omega)$, as defined in \cref{eq:labelingEq}, is the combination of two independent cost functions that are simultaneously minimized as a whole. The former term, $D_s(\omega_s)$, is referred to as the \emph{data} term, while the latter, $\sum_{r \in \mathcal{N}_{s}} V_{s,r}(\omega_s,\omega_r)$, is indistinctly referred to as the \emph{pairwise} or \emph{smoothing} term.
 The data term is the cost of assigning a particular label $l$ (also denoted $\omega_s$) to the site $s$ based on the image data of $s$, whereas the pairwise or smoothing term represents the cost of the assignation $\omega_s$ taking into account the labels of its neighbour sites, $\omega_r$, $r \in \mathcal{N}_{s}$. 

As illustrates \cref{fig:method}, in order to produce a segmentation, the image is \emph{mapped} or represented using \emph{super-pixels}. 

the elements here used 

 The image 
%
%according to a labelling set 
%
%Which have to be properly labelled and the appropiated label for each element must be infered from a training stage and 
%a discrete set of labels need to properly assigned to another discrete set of elemetns constituting the image.
%
%
\begin{figure}[htpb]
  \centering
  %\includegraphics[width=0.8\linewidth]{name.ext}
  \missingfigure{segmentation blocks}
  \caption{Conceptual block representation of the segmentation methodology}
  \label{fig:method}
\end{figure}

%%% Local Variables: 
%%% mode: latex
%%% TeX-master: "../../master.tex"
