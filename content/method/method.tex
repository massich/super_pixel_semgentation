% include the figures path relative to the master file
\graphicspath{ {./content/method/figures/} }

\section{Segmentation methodology description} 

Optimization methodologies offer a standardized manner to approach segmentation by minimizing an application-driven cost function~\cite{cremers2007review}.
\Cref{fig:method} illustrates a generic representation of the segmentation strategy here adopted to delineate breast tissues or lesions in \ac{us} images. 
The overall segmentation can be seen as a three steps strategy: 
(1) a mapping or encoding of the image into a discrete set of elements $\mathcal{S}$, 
(2) the optimization stage which is formulated as \emph{metric labelling} problem, 
and (3) re-mapping or re-coding of the labels obtained from the previous stage to produce the final delineation. 

\begin{figure}[htpb]
  \centering
  %\includegraphics[width=0.8\linewidth]{name.ext}
  \missingfigure{segmentation blocks}
  \caption{Conceptual block representation of the segmentation methodology}
  \label{fig:method}
\end{figure}

In order to formulate the segmentation like a metric labelling problem, the image is conceived as a discrete set of elements $\mathcal{S}$ that need to be labelled using a label $l$ from the labelling set $\mathcal{L}$ 
(i.e.\, $l \in \{\text{lesion}, \overline{\text{lesion}}\}$ 
or $l \in \{\text{lungs}, \text{fat},\,\cdots\,, \text{lesion}\}$).
Let $\mathcal{W}$ be all the possible labelling configurations of the set $\mathcal{S}$ given $\mathcal{L}$; and, let $U(\cdot)$ be a cost function encoding how good is a labelling configuration $\omega \in \mathcal{W}$ based on the appearance of the elements in $\mathcal{S}$, their inner relation and some designing constrains.
Then, the desired segmentation $\hat{\omega}$ corresponds to the labelling configuration that minimize this cost function, as described in \cref{eq:costMin}.
%minimizing this cost function, $\displaystyle \hat{\omega} = \arg \min_{\substack{\omega}} \,U(\omega)$. 

\begin{equation}
\hat{\omega} = \arg \min_{\substack{\omega}} \,U(\omega)
\label{eq:costmin}
\end{equation}

Selecting the appropriated strategy to minimize the cost function $U(\omega)$ is part of the designing process since depending on the nature of $U(\cdot)$ and $\mathcal{W}$, not all the minimizing strategies are suitable or desirable.

\Cref{eq:labelingeq} shows the details of the cost function which combines two independent cost.
Both costs are shaped by $\mathcal{s}$, evaluated in $\mathcal{W}$ and need to be simultaneously minimized as a whole.
The former term $D_s(\omega_s)$, is referred to as the \emph{data} term, while the latter, $\sum_{r \in \mathcal{N}_{s}} V_{s,r}(\omega_s,\omega_r)$, is indistinctly referred to as the \emph{pairwise} or \emph{smoothing} term.

\begin{equation}
  U(\omega) = \sum_{s\in s} D_s(\omega_s) + \sum_{s}\sum_{r \in \mathcal{N}_{s}} V_{s,r}(\omega_s,\omega_r)
  \label{eq:labelingeq}
\end{equation}

\Cref{fig:methodterms} takes the study case of delineating breast tissues in \ac{us} images, to relate \cref{eq:labelingeq,fig:method}, and offer an interpretation their terms and elements. 

Despite the fact that $\mathcal{s}$ can be any discrete set representing the image (i.e.\, pixels, overlapping or non overlapping windows, etc.), 
for this application, $\mathcal{s}$ is the super-pixels representation of the image~\cite{achanta2012slic}. 
%The super-pixels can be seen as the output of a over-segmentation process or as a set of pixel collections that are contiguous and coherent with respect to some metric. Either way super-pixels are no overlapped irregular groups of similar connected pixels~\cite{achanta2012slic}.
\Cref{fig:methodTerms:problem} shows a \ac{bus} image example and a its associated super-pixels representation $\mathcal{S}$ coloured according to the image's \ac{gt}.
Bear in mind, that the ultimate goal given an unseen \ac{bus} image is to represent the image as a set of super-pixels and infer the appropriated labelling for each of them.

\subsection{Data term} \label{sec:method:dataTerm}

Given a label configuration $\omega \in \mathcal{W}$, the data term penalizes the assignation of a particular label to a particular image element or site ($\omega_s = l$) based on the data associated to $s$. 
In this manner $D_s(\omega_s=l_\cmark) << D_s(\omega_s=l_\xmark)$. 
To perceive the effect or behaviour of this data term, \cref{fig:methodTerms:data} shows some labelling configurations $\omega^'$ where all the sites share the same label, $\omega' \in \{ \omega_s=l,~\forall s\in\mathcal{S}\}$

Designing $D(\cdot)$ that accomplish the desired behaviour by defining an obscure heuristic, is rather complicated. 
Therefore, an easier and cleaner approach is to take advantage of \ac{ml} techniques. 
The idea is to generate image or data model for each class, based on training samples, and let $D(\cdot)$ be a distance or goodness measure reflecting how likely is for the site $s$ to belong to class $l$.
\Cref{fig:method} shows how this is incorporated to the segmentation framework here proposed.
Each site $s$ is treated as a sample and the features to describe the site are extracted from the original image. 
For the work here reported, a \ac{svm} classifier is used to determine the data model during the training stage and during testing stage $D_s(\omega_s=l)$ corresponds to the distance between the testing sample and the vector supporting the data model associated to $l$. 

Notice that defining $D(\cdot)$ in this manner allows for many designing choices such as: which features to use, how is the class model created: which classifier, which training policy; or, how is defined the relation between the testing sample and the model.
Further discussion regarding the feature choices can be found in \cref{sec:featuers}, whereas other designing choices regarding \ac{ml} are out of the scope for this work.

\subsection{Pairwise or smoothing term} \label{sec:method:mrfTerm}
 
The pairwise term represents the cost of the assignation $\omega_s$ taking into account the labels of its neighbour sites, $\omega_r$, $r \in \mathcal{N}_{s}$. 

As illustrates \cref{fig:method}, in order to produce a segmentation, the image is \emph{mapped} or represented using \emph{super-pixels}. 

the elements here used 

 The image 
%
%according to a labelling set 
%
%Which have to be properly labelled and the appropiated label for each element must be infered from a training stage and 
%a discrete set of labels need to properly assigned to another discrete set of elemetns constituting the image.
%
%

%%% Local Variables: 
%%% mode: latex
%%% TeX-master: "../../master.tex"
