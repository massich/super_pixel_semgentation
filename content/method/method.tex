% include the figures path relative to the master file
\graphicspath{ {./content/method/figures/} }

\section{Segmentation methodology description} 

Optimization methodologies offer a standardized manner to approach segmentation by minimizing an application-driven cost function~\cite{cremers2007review}.
The segmentation strategy here adopted is formulated as a \emph{metric labelling} problem. 
The image is conceived as a discrete set of elements $\mathcal{S}$ that need to be labelled using a label $l$ from a labelling set $\mathcal{L}$ %(i.e.\, $l \in \text{\{lesion, lesion\}}$ or $l \in \text{\{lungs, fat, parenchyma, \cdots, lesion\}}$).
Let $\mathcal{W}$ be all the possible labelling configurations of the set $\mathcal{S}$ given $\mathcal{L}$, and let $U(\cdot)$ be a cost function encoding how good is a labelling configuration $\omega \in \mathcal{W}$ based on the appearance of the elements in $\mathcal{S}$, their relation and some designing constrains.
Then the desired segmentation $\hat{\omega}$, corresponds to the labelling configuration that minimizes this cost function $U(\cdot)$, further described in \Cref{eq:labelingEq}.

\begin{equation}
%  \displaystyle \hat{\omega} = \arg \min_{\substack{\omega}} \,U(\omega) \\
  U(\omega) = \sum_{s\in S} D_s(\omega_s) + \sum_{s}\sum_{r \in \mathcal{N}_{s}} V_{s,r}(\omega_s,\omega_r)
  \label{eq:labelingEq}
\end{equation} 


\Cref{fig:method} illustrates a generic version of the segmentation strategy here proposed. The image 
%
%according to a labelling set 
%
%Which have to be properly labelled and the appropiated label for each element must be infered from a training stage and 
%a discrete set of labels need to properly assigned to another discrete set of elemetns constituting the image.
%
%
\begin{figure}[htpb]
  \centering
  %\includegraphics[width=0.8\linewidth]{name.ext}
  \missingfigure{segmentation blocks}
  \caption{Conceptual block representation of the segmentation methodology}
  \label{fig:method}
\end{figure}

%%% Local Variables: 
%%% mode: latex
%%% TeX-master: "../../master.tex"
